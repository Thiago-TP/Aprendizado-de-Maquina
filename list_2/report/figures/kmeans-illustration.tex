\begin{tikzpicture}[> = latex]

    \def\r{.08}

    % Part 1: initialization
    \draw (1, 4) circle (.3) node {1};
    \drawDataset{black!10}{black!10}
    % Ceentroids
    \fill [red] (0.5, 1.2) circle (\r);
    \fill [blue] (1.6, 1.0) circle (\r);

    % Part 2: assigning samples according to distance
    \begin{scope}[shift={(4, 0)}]
        \draw (1, 4) circle (.3) node {2};
        \drawDataset{red!40}{blue!40}
        \draw [fill=red!40] (0.8, 0.6) circle (\r);
        % Old centroids
        \fill [red] (0.5, 1.2) circle (\r);
        \fill [blue] (1.6, 1.0) circle (\r);
        % New centroids
    \end{scope}

    % Part 3: calculate clusters' centroids 
    \begin{scope}[shift={(8, 0)}]
        \draw (1, 4) circle (.3) node {3};
        \drawDataset{red!40}{blue!40}
        \draw [fill=red!40] (0.8, 0.6) circle (\r);
        % Old centroids
        \fill [red] (0.5, 1.2) circle (\r) node (c1) {};
        \fill [blue] (1.6, 1.0) circle (\r) node (c2) {};

        % Centroids to be
        \draw (0.05, 1.625) circle (\r) node (c1-new) {};
        \draw (1.55, 0.583) circle (\r) node (c2-new) {};

        \draw [->] (c1.center) -- (c1-new.center);
        \draw [->] (c2.center) -- (c2-new.center);
    \end{scope}

    % Part 4: update centroids and clusters
    \begin{scope}[shift={(12, 0)}]
        \draw (1, 4) circle (.3) node {5};        
        \drawDataset{red!40}{blue!40}
        % Centroids
        \fill [red] (0.05, 1.625) circle (\r);
        \fill [blue] (1.55, 0.583) circle (\r);
    \end{scope}
\end{tikzpicture}